\documentclass[10pt,amsmath,amssymb,twocolumn,superscriptaddress,groupedaddress,nofootinbib,aps,prd,twocolumn]{revtex4-1}

\usepackage{graphicx}  % needed for figures
\graphicspath{{./plots/}}
\usepackage{bm}        % for math
\usepackage{enumitem}
\usepackage[colorlinks=true]{hyperref}

\pdfminorversion=7
\pdfsuppresswarningpagegroup=1

\providecommand{\texorpdfstring}[2]{#1}

\newcommand{\lr}[1]{\left(#1\right)}
\newcommand{\lrs}[1]{\left[#1\right]}
\newcommand{\vev}[1]{\left\langle #1 \right\rangle}
\newcommand{\ket}[1]{ \, | #1 \rangle }
\newcommand{\bra}[1]{ \langle #1 | \, }
\newcommand{\braket}[2]{ \langle #1 | #2 \rangle }
\newcommand{\mycomment}[1]{}
\renewcommand{\b}[1]{\mathbf{#1}}

\newcommand{\floor}[1]{\lfloor #1 \rfloor}

\begin{document}
\sloppy

\section{Introduction}
\label{sec:intro}

We consider the XXZ model Hamiltonian in random external magnetic field:
\begin{eqnarray}
\label{XXZ_Hamiltonian}
 \hat{H} = \hat{H}_0 + \hat{H}_I,
 \nonumber \\
 \hat{H}_0 = \sum\limits_{x,\alpha} \hat{\sigma}^{\alpha}_x \hat{\sigma}^{\alpha}_{x+1} ,
 \quad
 \hat{H}_I = \sum\limits_x \frac{h_x}{2} \hat{\sigma}^3_x ,
\end{eqnarray}
where $\hat{\sigma}^{\alpha}_x$ are the spin operators with the usual commutation relations $\lrs{\hat{\sigma}^{\alpha}_x, \hat{\sigma}^{\beta}_y} = i \delta_{x y} \epsilon^{\alpha\beta\gamma} \hat{\sigma}^{\gamma}$ and $h_x$ are coordinate-dependent random magnetic fields which take random values in the range $h_x \in \lrs{-h, h}$.

We impose periodic boundary conditions with length $L$: $x = 0 \ldots \lr{L-1}$, $x = \lr{L-1}+1$ is identified with $x = 0$. Since the Hamiltonian (\ref{XXZ_Hamiltonian}) commutes with the total spin in a fixed direction, we restrict our Hilbert space to the subspace of states with total spin zero in the direction $\alpha = 3$:
\begin{eqnarray}
\label{spin0_constraint}
 \sum\limits_{x} \hat{\sigma}^3_x \, \ket{\psi} = 0 .
\end{eqnarray}
The dimensionality of this subspace is $L!/\lr{\lr{L/2}!}^2$.

With these constraints, the model undergoes a many-body localization phase transition at $h \approx 3.75$ \cite{Luitz:1411.0660}, which can also be interpreted as an order-to-chaos phase transition \cite{Hanada:1803.08050}.

\begin{figure*}[h!pb]
  \centering
  \includegraphics[width=0.49\textwidth]{trotter_err_L10.pdf}
  \includegraphics[width=0.49\textwidth]{trotter_err_L12.pdf}\\
  \includegraphics[width=0.49\textwidth]{trotter_err_L14.pdf}
  \includegraphics[width=0.49\textwidth]{trotter_err_L16.pdf}\\
  \caption{Time dependence of the Trotter discretization error $\epsilon_T\lr{t, {\delta t}}/\lr{\delta t}^2$. We divide $\epsilon_T\lr{t, {\delta t}}$ by $\lr{\delta t}^2$ to demonstrate the expected scaling of the Trotter error as $\lr{\delta t}^2$.}
  \label{fig:trotter_err}
\end{figure*}

\section{Trotter discretization error}
\label{sec:trotter_err}

\begin{figure*}[h!pb]
  \centering
  \includegraphics[width=0.6\textwidth]{trotter_err_vs_L.pdf}\\
  \caption{Time dependence of the Trotter discretization error $\epsilon_T\lr{t, {\delta t}}$ for different lattice sizes.}
  \label{fig:trotter_err_vs_L}
\end{figure*}

We define the Trotter-discretized evolution operator as
\begin{eqnarray}
\label{trotter_evolution}
 U_T\lr{t, {\delta t}} = \lr{U_T\lr{\delta t}}^{\floor{\frac{t}{\delta t}}} ,
\end{eqnarray}
where $\floor{\ldots}$ is the floor function and $U_T\lr{\delta T}$ is the Trotter approximation to the evolution operator with the Trotter discretization time $\delta t$:
\begin{eqnarray}
\label{trotter_step}
 U_T\lr{\delta t} = e^{i \hat{H}_I \frac{\delta t}{2} } e^{i \hat{H}_0 {\delta t}} e^{i \hat{H}_I \frac{\delta t}{2} } .
\end{eqnarray}
This ``leapfrog'' form of the Trotter approximation ensures that the Trotter discretization error scales as ${\delta t}^2$.

In order to estimate the Trotter discretization error, we perform the time evolution of a random state vector $\ket{\psi\lr{0}}$ in the Hilbert subspace of the spin-0 states (\ref{spin0_constraint}) using the Trotter evolution operator (\ref{trotter_step}):
\begin{eqnarray}
\label{trotter_state_evolution}
 \ket{\psi\lr{t, {\delta t}}} = U_T\lr{t, {\delta t}} \ket{\psi\lr{0}} .
\end{eqnarray}
We define the Trotter discretization error $\epsilon_T\lr{t, {\delta t}}$ as the norm of the difference between the state vectors $\ket{\psi\lr{t, {\delta t}}}$ and $\ket{\psi\lr{t, {\delta t}/2}}$ evolved with Trotter time steps ${\delta t}$ and ${\delta t}/2$:
\begin{eqnarray}
\label{trotter_err_def}
 \epsilon_T\lr{t, {\delta t}} = \vev{|| \ket{\psi\lr{t, {\delta t}}} - \ket{\psi\lr{t, {\delta t}/2}} || } ,
\end{eqnarray}
where $\vev{\ldots}$ denotes averaging over random initial state vectors $\ket{\psi\lr{0}}$ and random samples of the random magnetic field $h_x$ in the Hamiltonian (\ref{XXZ_Hamiltonian}) and the norm is defined as $|| \ket{\phi} || = \sqrt{\braket{\phi}{\phi}}$.

Since in our numerical simulation the Trotter evolution is also subject to numerical roundoff errors, we also control the deviation of $U_T\lr{t, {\delta t}}$ from unitarity in terms of the deviation of the norms of $\ket{\psi\lr{t, {\delta t}}}$ and $\ket{\psi\lr{t, {\delta t}/2}}$ from unity:
\begin{eqnarray}
\label{norm_err_def}
 \epsilon_U =
 \nonumber \\ =
  \vev{ \max\lr{
  \left| \, ||\ket{\psi\lr{t, {\delta t}}}|| - 1 \right|, 
  \left| \, ||\ket{\psi\lr{t, {\delta t}/2}}|| - 1 \right|
 } } .
\end{eqnarray}
Figures~\ref{fig:norm_err_logscale} and \ref{fig:trotter_err_superlong} (plot on the right) show that unitarity error $\epsilon_U$ appears to be reasonably small up to the longest evolution time $t = 10^4$ which we've considered.

\section{Numerical results}
\label{sec:numres}

On Fig.~\ref{fig:trotter_err} we show the time dependence of the Trotter discretization error $\epsilon_T\lr{t, {\delta t}}/\lr{\delta t}^2$ for different sizes of the spin chain: $L = 10, 12, 14, 16$, and at two different values $h = 1$ and $h = 6$ of the random magnetic field. Dividing by $\lr{\delta t}^2$ and comparing the data for two different time steps, we see that the Trotter error exhibits the expected quadratic scaling in the discretization step $\delta t$ at all times. As Fig.~\ref{fig:trotter_err_vs_L} shows, there appear to be no qualitative differences in the behavior of $\epsilon_T\lr{t, \delta t}$ for different lattice sizes.

At early times $t \lesssim 1$, the time dependence of $\epsilon_T\lr{t, \delta t}$ appears to be quite different in the ordered phase at $h = 1$ and in the chaotic MBL phase at $h = 6$. Notably, in the chaotic phase the system exhibits some oscillations of the Trotter error resembling quasinormal ringing. However, as one can see from Fig.~\ref{fig:trotter_err_superlong}, $\epsilon_T\lr{t, \delta t}$ \textbf{grows linearly} with $t$ at later times $ 1 \lesssim t \lesssim 2 \cdot 10^3$ (for ${\delta t} = 5 \cdot 10^{-2}$) both for $h=1$ and $h=6$. At even later times, the Trotter error $\epsilon_T\lr{t, \delta t}$ defined in (\ref{trotter_err_def}) approaches the limiting value $\epsilon_T\lr{t, \delta t} = \sqrt{2}$. When the Trotter error reaches this limit, discretized evolution operators $U_T\lr{t, {\delta t}}$ and $U_T\lr{t, {\delta t}/2}$ become completely uncorrelated, and the difference $\ket{\psi\lr{t, {\delta t}}} - \ket{\psi\lr{t, {\delta t}/2}}$ behaves as the difference of two independent random state vectors uniformly distributed over unit spheres.

We have also checked that one obtains very similar results for other possible definitions of the Trotter error: 
\begin{itemize}
 \item $\epsilon_T$ in terms of the difference of two state vectors as in (\ref{trotter_err_def}), but with either the ground-state vector of $\hat{H}_0$, or another eigenstate in the bulk of the spectrum instead of the random state vector $\ket{\psi}$. Interestingly, deviations from the random-state appear to be strongest for the ground state of $\hat{H}_0$
 \item $\epsilon_T = || \hat{U}_T\lr{t, {\delta} t} - \hat{U}_T\lr{t, {\delta} t/2} ||$, where $||\ldots||$ is the Frobenius norm.
\end{itemize}

\begin{figure*}[h!pb]
  \centering
  \includegraphics[width=0.49\textwidth]{trotter_err_L12_superlong.pdf}
  \includegraphics[width=0.49\textwidth]{norm_err_L12_superlong.pdf}\\
  \caption{Trotter discretization error $\epsilon_T\lr{t, {\delta t}}$ (on the left) and unitarity error $\epsilon_U\lr{t, {\delta} t}$ for $L=12$ on a large time scale (up to $t = 10^4$).}
  \label{fig:trotter_err_superlong}
\end{figure*}

\begin{figure*}[h!pb]
  \centering
  \includegraphics[width=0.49\textwidth]{trotter_err_L10_log.pdf}
  \includegraphics[width=0.49\textwidth]{trotter_err_L12_log.pdf}\\
  \includegraphics[width=0.49\textwidth]{trotter_err_L14_log.pdf}
  \includegraphics[width=0.49\textwidth]{trotter_err_L16_log.pdf}\\
  \caption{Time dependence of the Trotter discretization error $\epsilon_T\lr{t, {\delta t}}$ in logarithmic scale.}
  \label{fig:trotter_err_logscale}
\end{figure*}

\begin{figure*}[h!pb]
  \centering
  \includegraphics[width=0.49\textwidth]{norm_err_L10_log.pdf}
  \includegraphics[width=0.49\textwidth]{norm_err_L12_log.pdf}\\
  \includegraphics[width=0.49\textwidth]{norm_err_L14_log.pdf}
  \includegraphics[width=0.49\textwidth]{norm_err_L16_log.pdf}\\
  \caption{Time dependence of the unitarity error $\epsilon_U\lr{t, {\delta t}}$ in logarithmic scale.}
  \label{fig:norm_err_logscale}
\end{figure*}

\bibliographystyle{apsrev4-1}
\bibliography{Buividovich}

\end{document} 